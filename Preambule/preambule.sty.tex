\newcommand{\HRule}{\rule{\linewidth}{0.5mm}}
\usepackage[utf8]{inputenc}
\usepackage[french,english]{babel}
\usepackage[T1]{fontenc}
\usepackage{amsmath,amsthm,amsfonts,amssymb,mathtools}
\newtheorem{definition}{Définition}
\usepackage{graphicx}
\usepackage{adjustbox}
\usepackage{subcaption}
\usepackage{float}
\usepackage{xcolor,color}
\definecolor{silver}{rgb}{0.75, 0.75, 0.75}
\usepackage{tikz}

\let\clipbox\relax
\usetikzlibrary{shapes.geometric}
\usepackage{hyperref}
%\usepackage[addtotoc]{abstract}
\usepackage[left=2cm,right=2cm,top=2cm,bottom=2cm]{geometry}
\usepackage{fancyhdr}
\pagestyle{fancy}
\usepackage{url}
\usepackage{varioref}
\usepackage{cleveref}
\usepackage{hyperref}
\usepackage{nameref}
\usepackage{titleref}
\usepackage{lettrine}
\usepackage{tabto}
\usepackage{lipsum}
\usepackage{siunitx} % Required for alignment
\usepackage{multirow}
\usepackage{longtable}
\usepackage{booktabs}
\usepackage{diagbox}
%\usepackage{tikz-uml}
%\usepackage{pgf-umlcd}
%\usepackage{uml}
%\usepackage[table]{xcolor}
%\usepackage{pgfplots}
\sisetup{
	round-mode          = places, % Rounds numbers
	round-precision     = 2, % to 2 places
}

\usepackage{listings} % Pour utiliser la commande lstset et lstdefinestyle

\lstset{
	basicstyle=\tiny,
	%columns=1,
	breaklines=true
}
\lstdefinestyle{javastyle}{
	%language=java
	basicstyle=\ttfamily\tiny, % Police de caractères et taille
	numbers=left, % Affichage des numéros de lignes à gauche
	numberstyle=\tiny\color{gray}, % Style des numéros de lignes (couleur grise et taille petite)
	backgroundcolor=\colorbox{gray!20}, % Fond gris foncé
	frame=single, % Cadre autour du code
	framesep=2pt, % Marge entre le code et le cadre
}

\lstdefinestyle{monstyle}{
	language=Java,
	frame=single,
	backgroundcolor=\color{lightgray},
	numbers=left
}


\usepackage[
type={CC},
modifier={by-nc-sa},
version={4.0},
]{doclicense}
%header
\renewcommand{\headrulewidth}{1pt}
\fancyhead[C]{} 
\fancyhead[L]{\leftmark}
\fancyhead[R]{\rightmark}
%footer
\renewcommand{\footrulewidth}{1pt}
\fancyfoot[C]{\textbf{page \thepage}} 
\fancyfoot[L]{\doclicenseNameRef
	— \copyright Baptiste Grosjean}
\fancyfoot[R]{Ann\'eee acad\'emique 2022-2023}
%bibilography
\usepackage[backend=biber,natbib=true,sortcites=true,style=numeric,sorting=none]{biblatex}
\usepackage[babel=true,autostyle=true]{csquotes}
%\renewcommand{\thesection}{\Roman{section}}
\author{Baptiste Grosjean}
\title{}
\date{\today}
